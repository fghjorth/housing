\documentclass[12pt,a4paper]{article}
\usepackage[utf8]{inputenc}
\usepackage{amsmath}
\usepackage{amsfonts}
\usepackage{amssymb}
\usepackage{graphicx}
\usepackage{booktabs}
\usepackage{natbib}
\usepackage{dcolumn}
\usepackage{setspace}
\usepackage{array}
\usepackage{pdflscape} %allows for rotating pages with wide tables
\newcolumntype{P}[1]{>{\raggedright\arraybackslash}p{#1}}
%\usepackage{tabulary}
\usepackage[T1]{fontenc}
\usepackage{lmodern}
\usepackage{multirow}
\usepackage{multicol}
%\usepackage{mathptmx} %times font
%\usepackage{tgtermes} %times font
\usepackage[protrusion=true,expansion=true]{microtype}
\usepackage[top=1in, bottom=1in, left=1in, right=1in]{geometry}
\usepackage{hyperref}
\usepackage{color,soul} %highlighting
\usepackage[font={footnotesize}]{caption}

%\usepackage{endnotes}
%\usepackage[heads,nolists,tablesfirst]{endfloat} %places tables and figures at the end
%

\title{ Falling house prices hurt incumbents  }
%What goes with whom? Group-centric policy attitudes and the role of casual observation
%Context effects and ethnicized policy attitudes}
%Contextual Ethnic Diversity \\ and Ethnicized Crime Attitudes
%OR: Issue evolution and contextual learning?

\author{
Martin V. Larsen \\ Frederik Hjorth}

%, \textt{fh@ifs.ku.dk}, (+45) 26 27 24 41 }  } 


\begin{document}

\maketitle

\begin{center}
\textsc{draft - please do not quote, cite, or circulate}
\end{center}

\begin{abstract}
\noindent In this paper we examine the extent to which house prices shape incumbent support in Danish elections. We link detailed data on local house prices to two different datasets of electoral behavior;  (1)  a population-based dataset on voting behavior at the precinct level and (2) a set of surveys on vote intentions. Across both datasets we find that house prices influence incumbent support asymmetrically, such that voters punish incumbents only when prices are falling. That is, voters seem to have a strong loss aversion in how they attribute responsibility for changes in housing prices. The findings hold important implications for the role personal economic grievances in voter behavior and for the political incentives faced by policy-makers.
\end{abstract}

%\begin{keyword}
%\doublespacing
%%x \sep y
%\end{keyword}

%\end{frontmatter}

\newpage

\onehalfspacing
%\doublespacing

\section{Introduction}


\clearpage

\singlespacing

\bibliographystyle{../../bibliography/model2-names}
\bibliography{../../bibliography/library}

\end{document}
